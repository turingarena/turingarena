\documentclass[a4paper,11pt]{article}
\usepackage{lmodern}
\renewcommand*\familydefault{\sfdefault}
\usepackage{sfmath}
\usepackage[utf8]{inputenc}
\usepackage[T1]{fontenc}
\usepackage[italian]{babel}
\usepackage{indentfirst}
\usepackage{graphicx}
%\usepackage[group-separator={\,}]{siunitx}
\usepackage[left=2cm, right=2cm, bottom=3cm]{geometry}
\frenchspacing

\newcommand{\num}[1]{#1}

% Macro varie...
\newcommand{\file}[1]{\texttt{#1}}
\renewcommand{\arraystretch}{1.3}
\newcommand{\esempio}[2]{
\noindent\begin{minipage}{\textwidth}
\begin{tabular}{|p{11cm}|p{5cm}|}
    \hline
    \textbf{File \file{input.txt}} & \textbf{File \file{output.txt}}\\
    \hline
    \tt \small #1 &
    \tt \small #2 \\
    \hline
\end{tabular}
\end{minipage}
}

\pagestyle{empty}

% Dati del task
\newcommand{\nome}{Trova la somma pari massima v2.0}
\newcommand{\nomebreve}{easy3}

\begin{document}
% Intestazione
\noindent{\Huge \textbf \nome~(\texttt{\nomebreve})}

% Descrizione del task
\section*{Descrizione del problema}
Topolino ha ricevuto in regalo un'altra sequenza di $N$ numeri interi $S_i$. Vuole cercare,
per ogni possibile coppia di numeri (in posizioni distinte) scelti dalla sequenza, la somma
massima che sia anche pari.
Aiuta Topolino scrivendo un programma che, presi in ingresso $N$ e la sequenza
di interi $S_i$, stampi in uscita la somma massima di una coppia \textit{tra le somme pari}.

Se $N$ fosse uguale a 10 e la sequenza ricevuta da Topolino fosse la seguente:

\begin{center}
\setlength{\tabcolsep}{10pt}
    \begin{tabular}{| c | c | c | c | c | c | c | c | c | c | c | c |}
    \hline
    1 & 2 & 3 & 4 & 5 & 6 & 7 & 8 & 9 & 10 \\
    \hline
    \end{tabular}
\end{center}

\noindent
allora il tuo programma dovrebbe stampare \textbf{18} (ottenibile come 10 + 8).

% input.txt
\section*{Dati di input}
\noindent
Nel file \file{input.txt} sono presenti due righe di testo: nella prima c'è un singolo
numero intero positivo $N$, nella seconda ci sono gli $N$ interi $S_i$, separati da spazio,
che compongono la sequenza di Topolino.

% output.txt
\section*{Dati di output}
\noindent
Nel file \file{output.txt} dovrai stampare un singolo numero intero, la somma massima tra
le somme pari delle coppie ricevute in input. \textbf{Se tale somma non dovesse esistere,
allora dovrai stampare \texttt{-1}}.

% Assunzioni
\section*{Assunzioni}
\begin{itemize}
\item $1 \le N \le 100\,000$.
\item $0 \le S_i < 1\,000\,000$.
\end{itemize}

% Esempi
\section*{Esempi di input/output}
\setlength{\tabcolsep}{6pt}
\esempio{
10

1 2 3 4 5 6 7 8 9 10
}{18}

\esempio{
2

13 13
}{26}

\end{document}

