\documentclass[a4paper,11pt]{article}
\usepackage{lmodern}
\renewcommand*\familydefault{\sfdefault}
\usepackage{sfmath}
\usepackage[utf8]{inputenc}
\usepackage[T1]{fontenc}
\usepackage[italian]{babel}
\usepackage{indentfirst}
\usepackage{graphicx}
%\usepackage[group-separator={\,}]{siunitx}
\usepackage[left=2cm, right=2cm, bottom=3cm]{geometry}
\frenchspacing

\newcommand{\num}[1]{#1}

% Macro varie...
\newcommand{\file}[1]{\texttt{#1}}
\renewcommand{\arraystretch}{1.3}
\newcommand{\esempio}[2]{
\noindent\begin{minipage}{\textwidth}
\begin{tabular}{|p{11cm}|p{5cm}|}
    \hline
    \textbf{File \file{input.txt}} & \textbf{File \file{output.txt}}\\
    \hline
    \tt \small #1 &
    \tt \small #2 \\
    \hline
\end{tabular}
\end{minipage}
}

\pagestyle{empty}

% Dati del task
\newcommand{\nome}{Trova la somma pari massima}
\newcommand{\nomebreve}{easy2}

\begin{document}
% Intestazione
\noindent{\Huge \textbf \nome~(\texttt{\nomebreve})}

% Descrizione del task
\section*{Descrizione del problema}
Topolino ha ricevuto in regalo una sequenza di $N$ coppie $(a_i,b_i)$ di numeri
interi. Ha deciso che per ogni coppia calcolerà la somma $a_i+b_i$, con
l'obiettivo di cercare la somma più grande. Fin qui tutto normale, purtroppo però
il dottore ha ordinato a Topolino di stare alla larga dai numeri dispari!

Aiuta Topolino scrivendo un programma che, presi in ingresso $N$ e la sequenza
di coppie $(a_i, b_i)$, stampi in uscita la somma massima \textit{tra quelle pari}.

Se $N$ fosse uguale a 5 e la sequenza di coppie ricevuta da Topolino fosse
la seguente:

\begin{center}
\setlength{\tabcolsep}{10pt}
    \begin{tabular}{| c | c | c | c | c | c |}
    \hline
    \textbf{coppie:} & (746, 985) & (168, 440) & (425, 940) & (72, 376) & (801, 264) \\ \hline
    \textbf{somme:} & 1731 & 608 & 1365 & 448 & 1065 \\
    \hline
    \end{tabular}
\end{center}

\noindent
allora il tuo programma dovrebbe stampare \textbf{608}.

% input.txt
\section*{Dati di input}
\noindent
Nel file \file{input.txt} sono presenti $N+1$ righe di testo: nella prima c'è un singolo
numero intero positivo $N$, dalla seconda riga alla $N+1$-esima ci sono le $N$ coppie di
interi $a_i$ e $b_i$, separati da uno spazio, che compongono le coppie $(a_i,b_i)$ nella
sequenza di Topolino.

% output.txt
\section*{Dati di output}
\noindent
Nel file \file{output.txt} dovrai stampare un singolo numero intero, la somma massima tra
le somme pari delle coppie ricevute in input. \textbf{Se tale somma non dovesse esistere,
allora dovrai stampare \texttt{-1}}.

% Assunzioni
\section*{Assunzioni}
\begin{itemize}
\item $1 \le N \le 1000$.
\item $0 \le a_i, b_i < 1000$.
\end{itemize}

% Esempi
\section*{Esempi di input/output}
\setlength{\tabcolsep}{6pt}
\esempio{
5

746 985

168 440

425 940

72 376

801 264
}{608}

\esempio{
1

383 886
}{-1}

\end{document}
