\documentclass[a4paper,11pt]{article}
\usepackage{lmodern}
\renewcommand*\familydefault{\sfdefault}
\usepackage{sfmath}
\usepackage[utf8]{inputenc}
\usepackage[T1]{fontenc}
\usepackage[italian]{babel}
\usepackage{indentfirst}
\usepackage{graphicx}
%\usepackage[group-separator={\,}]{siunitx}
\usepackage[left=2cm, right=2cm, bottom=3cm]{geometry}
\frenchspacing

\newcommand{\num}[1]{#1}

% Macro varie...
\newcommand{\file}[1]{\texttt{#1}}
\renewcommand{\arraystretch}{1.3}
\newcommand{\esempio}[2]{
\noindent\begin{minipage}{\textwidth}
\begin{tabular}{|p{11cm}|p{5cm}|}
    \hline
    \textbf{File \file{input.txt}} & \textbf{File \file{output.txt}}\\
    \hline
    \tt \small #1 &
    \tt \small #2 \\
    \hline
\end{tabular}
\end{minipage}
}

\pagestyle{empty}

% Dati del task
\newcommand{\nome}{Trova il massimo}
\newcommand{\nomebreve}{easy1}

\begin{document}
% Intestazione
\noindent{\Huge \textbf \nome~(\texttt{\nomebreve})}

% Descrizione del task
\section*{Descrizione del problema}
Topolino ha ricevuto in regalo una sequenza di $N$ numeri interi. Puoi aiutarlo
a trovare il numero più grande presente nella sequenza scrivendo un programma?
Se $N$ fosse uguale a 12 e la sequenza ricevuta da Topolino fosse la seguente:

\begin{center}
\setlength{\tabcolsep}{10pt}
    \begin{tabular}{| c | c | c | c | c | c | c | c | c | c | c | c |}
    \hline
    -331 & -341 & 389 & 349 & -37 & -287 & 441 & -871 & -913 & -853 & -617 & -150 \\
    \hline
    \end{tabular}
\end{center}

\noindent
allora il tuo programma dovrebbe restituire \textbf{441}.

% input.txt
\section*{Dati di input}
\noindent
Nel file \file{input.txt} sono presenti due righe di testo: nella prima c'è un singolo
numero intero positivo $N$; nella seconda riga ci sono gli $N$ interi $S_i$ che
compongono la sequenza di Topolino, separati da spazio.

% output.txt
\section*{Dati di output}
\noindent
Nel file \file{output.txt} dovrai stampare un singolo numero intero, il valore massimo
della sequenza.

% Assunzioni
\section*{Assunzioni}
\begin{itemize}
\item $1 \le N \le 1000$.
\item $|S_i| < 1000$, ovvero $-1000 < S_i < 1000$.
\end{itemize}

% Esempi
\section*{Esempi di input/output}
\setlength{\tabcolsep}{6pt}
\esempio{
12

-331 -341 389 349 -37 -287 441 -871 -913 -853 -617 -150
}{441}

\esempio{
3

896 -242 -311
}{896}

\esempio{
1

-677
}{-677}

\end{document}
